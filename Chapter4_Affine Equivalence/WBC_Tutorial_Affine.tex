\documentclass{beamer}
\usepackage{times}
\usepackage{tikz}
\usepackage{beamerthemesplit}
\usepackage{tcolorbox}

\title{A Tutorial of White-Box Cryptography \\ Chapter 4 Affine Equivalence}
\author{Zheng Gong\inst{1,2}\\ \url{cis.gong@gmail.com}}
\institute{\inst{1}{School of Computer Science, South China Normal University} \\ \inst{2}{Mobile Applications And Security Engineering Center of Guangdong Province}}

\date{\today}

\begin{document}

\frame
{
 \titlepage
}

\section[Outline]{}
\frame{\tableofcontents}

\section{Linear and Affine Equivalences for WBC Cryptanalysis}
\subsection{A Toolbox for Cryptanalysis: Linear and Affine Equivalence Algorithms}
\frame
{
  \frametitle{Linear and Affine Equivalence of Sboxes}

\begin{itemize}
\item At Eurocrypt 2003, Biryukov \textit{et al.} proposed a cryptanalysis toolkit for analyzing linear and affine equivalence of sboxes.
\item The toolkit can be used to extraction of the WBC look-up table $A\circ S \circ B \rightarrow S$.
\end{itemize}

\begin{center}
\begin{tikzpicture}
    \node[anchor=south west,inner sep=0] (image) at (0,0) { \includegraphics[width=7cm, height=3.5cm]{./pics/LinearAffine_EC2003.png}};

    %\begin{scope}[x={(image.south east)},y={(image.north west)}]
        %\draw[help lines,xstep=.1,ystep=.1] (0,0) grid (1,1);
        %\foreach \x in {0,1,...,9} { \node [anchor=north] at (\x/10,0) {0.\x}; }
        %\foreach \y in {0,1,...,9} { \node [anchor=east] at (0,\y/10) {0.\y}; }
        %\draw[green, ultra thick, rounded corners] (0.24,0.18) rectangle (0.50,0.32);
    %\end{scope}
\end{tikzpicture}
\end{center}
}


\frame
{
\begin{center}
\textbf{Thanks for your attentions!}
\end{center}
\begin{center}
\begin{tikzpicture}
    \node[anchor=south west,inner sep=0] (image) at (0,0) { \includegraphics[width=4cm, height=4cm]{./pics/WBC_BG.png}};

    %\begin{scope}[x={(image.south east)},y={(image.north west)}]
        %\draw[help lines,xstep=.1,ystep=.1] (0,0) grid (1,1);
        %\foreach \x in {0,1,...,9} { \node [anchor=north] at (\x/10,0) {0.\x}; }
        %\foreach \y in {0,1,...,9} { \node [anchor=east] at (0,\y/10) {0.\y}; }
        %\draw[green, ultra thick, rounded corners] (0.24,0.18) rectangle (0.50,0.32);
    %\end{scope}
\end{tikzpicture}

\end{center}
}

\end{document}
